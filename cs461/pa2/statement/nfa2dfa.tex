\documentclass[11pt]{article}
\usepackage{graphicx}
\usepackage[margin=1.0in]{geometry}

\renewcommand{\baselinestretch}{1}


\begin{document}

\begin{center}
  \large{COSC -- Fall 2014 \\
    Assignment 1 \\
    nfa2dfa -- Subset construction to convert NFA to DFA
  }
\end{center}

Your assignment is to implement the subset construction algorithm (Figure
3.32 in the text) to convert an NFA to the equivalent DFA.
You will also need to implement other algorithms, such as those for the
\emph{move} (Figure 3.31) and \emph{$\epsilon$--closure} (Figure 3.33)
operations, required for subset construction.

Both the input NFA and the output DFA will be represented as a transition
table.
You should represent the NFA and DFA states as integer numbers, starting at
1.
A special symbol 'E' will be used to indicate the $\epsilon$--transition.
You may choose any language you wish to implement the assignment.
The input will come from \emph{standard input} and the output should be
printed to \emph{standard output}.
You should not make any assumptions regarding the maximum number of states or
transitions in the input NFA or the output DFA.
You should also carefully consider how you will represent the NFA and
corresponding DFA in memory.
If you struggle writing the algorithm, consider using a different data
structure.

After you complete your implementation, you should rigorously test your
program to ensure its correctness. We will test your code using the inputs
available from the course website as well as a number of other randomly
generated input NFA's.

Finally, you will submit a project report. The project report is a short (one to two
page, single-spaced) document that describes:

\begin{enumerate}
\item (in your own words) the problem you set set out to solve, 
\item your approach (i.e. design and relevant implementation details) for
solving this problem,
\item how you debugged and tested your solution, and 
\item any issues you had in completing the assignment.
\end{enumerate}

This assignment will be due on \textbf{Tuesday, September 16}.
You should comment your program so that others (e.g. the grader) can
understand it. You should also have comments at the top of the file
indicating your name, this course, and the assignment.
You should attempt to match your output as closely to my output as possible.
E-mail a gzipped tar file (created with \texttt{tar xvzf ...}) with your
source code, instructions for building and running your program, and a pdf of
your report to \texttt{teffler@vols.utk.edu} and CC it to
\texttt{mrjantz@utk.edu} before the beginning of class on Tuesday, September
16$^{th}$.

Partial credits will be given for incomplete efforts. However, a program
that does not compile or run will get 0 points. Point breakdown is below:

\begin{itemize}
\item Reads / writes from standard input and standard output            (10)
\item Input file parsed into NFA data structure                         (10)
\item NFA to DFA conversion                                             (40)
\item Output is clear and in the appropriate format                     (10)
\item Code is easy to understand (e.g., contains appropriate comments)  (10)
\item Project report                                                    (20)
\end{itemize}

\newpage

\noindent
Example input NFA:
\begin{verbatim}
Initial State:  1
Final States:   {11}
Total States:   11
State   a                b                E
1       {}               {}               {2,5}
2       {3}              {}               {}
3       {}               {4}              {}
4       {}               {}               {8}
5       {}               {6}              {}
6       {7}              {}               {}
7       {}               {}               {8}
8       {}               {}               {9,11}
9       {10}             {}               {}
10      {}               {}               {9,11}
11      {}               {}               {}
\end{verbatim}

\noindent
Output from NFA2DFA executable:
\begin{verbatim}
reading NFA ... done.

creating corresponding DFA ...
new DFA state:  1    -->  {1,2,5}
new DFA state:  2    -->  {3}
new DFA state:  3    -->  {6}
new DFA state:  4    -->  {4,8,9,11}
new DFA state:  5    -->  {7,8,9,11}
new DFA state:  6    -->  {9,10,11}
done.

final DFA:
Initial State:  1
Final States:   {4,5,6}
Total States:   6
State   a                b
1       {2}              {3}
2       {}               {4}
3       {5}              {}
4       {6}              {}
5       {6}              {}
6       {6}              {}
\end{verbatim}

\end{document}
